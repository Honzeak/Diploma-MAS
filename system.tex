\documentclass[main.tex]{subfiles}
\graphicspath{{./img}}

\begin{document}
    
\section{Proposed system}\label{sec-system}

This section is dedicated to designing the system that will be used to implement ITS in the
vehicle simulator software. A framework dedicated for generic MAS-based ITS system
implementation will be designed, utilizing the principles and paradigms gathered in the
preceding sections (\ref{its}, \ref{mas}). Firstly, the actors (agents) within the systems will
be defined, i.e. their features, characteristics, state variables and their goals, as well as
interfaces to the environment. This will form the elementary foundation that will be used to
build an actual system.  As a follow up, the organizational structure will be discussed and
proposed. The objective will be to define an organizational structure so that it will
facilitate the aspect of control and configuration of the system. The last important step will
be to define interactions between agents.  This will for example include knowledge sharing,
conflict resolution and overall communication interface proposal, including shared vocabulary
and communication layers definition. These steps should, in the end, lead to a full system 
specification that will serve as a framework to implement agent-based ITSs in an interactive 
vehicle simulator.

\subsection{Agent architecture}

As per the previous section(s), where the individual MAS architectures have been reviewed
(section \ref{mas}), it was decided to utilize \emph{hybrid architecture} (section
\ref{hybrid-arch}), which will offer sufficient flexibility. Such modeling flexibility is 
needed primarily because there won't be a single, concrete system to model, but rather a 
generic system that will facilitate arbitrary agent-based ITS implementation. As such, it 
makes sense to choose a hybrid architecture, which will ensure there will be optimal balance 
between robust, reactive behaviour without giving up capabilities to model complex behaviour.


\clearpage

\end{document}