\documentclass[main.tex]{subfiles}
\graphicspath{{./img}}

\begin{document}
    
\section{Proposed system}\label{sec-system}

This section is dedicated to designing the system that will be used to implement ITS in the
vehicle simulator software. A framework dedicated for generic MAS-based ITS system
implementation will be designed, utilizing the principles and paradigms gathered in the
preceding sections (\ref{its}, \ref{mas}). Firstly, the \emph{micro-architecture} will be
defined, i.e. the specification of the system's actors (agents) - their features,
characteristics, state variables and their goals, as well as interfaces to the environment.
This will form the elementary foundation that will be used to build an actual system. As a
follow up, agent interaction interface will be defined. This will for example include
knowledge sharing, conflict resolution and overall communication interface proposal, including
shared vocabulary and communication layers definition. The interaction interface will,
subsequently, lead to the \emph{macro-architecture} proposal. The structure of the system and
its behaviour will be defined. This will, among other topics, include discussing the
organizational and functional structure. The motivation behind functional structure definition
will be to define the behaviorist capabilities of the system and its configuration. The main
underlying motivation behind the organizational structure definition will be to facilitate the
aspect of control of the system. These steps should, in the end, lead to a full system
specification that will serve as a framework to implement agent-based ITSs in an interactive
vehicle simulator.

\subsection{Agent architecture}

As per the previous section(s), where the individual MAS architectures have been reviewed
(section \ref{mas}), it was decided to utilize \emph{hybrid architecture} (section
\ref{hybrid-arch}), which will offer sufficient flexibility. Such modeling flexibility is 
needed primarily because there won't be a single, concrete system to model, but rather a 
generic system that will facilitate arbitrary agent-based ITS implementation. As such, it 
makes sense to choose a hybrid architecture, which will ensure there will be optimal balance 
between robust, reactive behaviour without giving up capabilities to model complex behaviour.

The elementary characteristics that were defined according to \cite{ParasumannaGokulan2010}
must shall also be taken into account: 

\begin{itemize}
    \item Situatedness
    \item Autonomy 
    \item Inferential capability 
    \item Social behaviour 
\end{itemize}



\clearpage

\end{document}