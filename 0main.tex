\documentclass[titlepage, 12pt]{article}

\title{
\includegraphics[width=.3\textwidth]{cvut-logo.jpg}\par
\vspace{10mm}
\indent
\textbf{CZECH TECHNICAL UNIVERSITY IN PRAGUE}

FACULTY OF TRANSPORT SCIENCES

\vfill

{\Large JAN MACEK}
\vspace{10mm}

AGENT-BASED MODELING OF ITS SYSTEMS IN VEHICLE SIMULATOR 
\vspace{15mm}

{\Large MASTER'S THESIS}
\vfill

}
\date{\Large 2022}

\usepackage[utf8]{inputenc}
\usepackage[T1]{fontenc}
% \usepackage{cm-super}
\usepackage{listings}
\usepackage{fancyhdr}
\usepackage{graphicx}
\usepackage{parskip}
\usepackage{svg}
\usepackage[a4paper, top=25mm, bottom=25mm, left=30mm, right=20mm, twoside]{geometry}
\usepackage{tabularx}
\usepackage{caption}
\usepackage{hyperref}
\usepackage{xcolor}
\usepackage{amsmath}
\usepackage[sorting=none]{biblatex} 
\usepackage{color}
\usepackage{pdfpages}
\usepackage{enumitem}
\usepackage{amssymb}
\usepackage{booktabs}
\usepackage{microtype}
\usepackage{subfiles}
\usepackage{multirow}
\usepackage{pdfpages}
\usepackage{enumitem}
\usepackage{soul}

\definecolor{dkgreen}{rgb}{0,0.6,0}
\definecolor{gray}{rgb}{0.5,0.5,0.5}
\definecolor{mauve}{rgb}{0.58,0,0.82}
\definecolor{inline}{rgb}{0.25,0.25,0.80}


\addbibresource{references.bib}

\hypersetup{
    colorlinks,
    linkcolor={red!50!black},
    citecolor={blue!50!black},
    urlcolor={blue!70!black}
}

\pagestyle{fancy}

\graphicspath{{./img}}

\fancyhead[L]{Jan Macek}
\setlength{\headheight}{15pt}

\newlength{\varSepscale}
\setlength{\varSepscale}{16pt}
\newcommand{\itemspacing}[1]{\setlength\itemsep{\dimexpr #1\varSepscale-\varSepscale}}
\newcommand{\term}[1]{\hspace*{2em}\textit{#1}\, -\,}
\newcommand{\rot}[1]{\rotatebox[origin=c]{90}{#1}}
\newcommand{\talign}[3]{\makebox[#2em]{#1\hfill}#3}

%DOCUMENT
\begin{document}
\setlength{\baselineskip}{1.5em}
\maketitle

\includepdf[pages=-]{zadani-dp-jan-macek.pdf}

\begin{abstract}
    The use of Intelligent Transport Systems (ITS) has become increasingly important in
    managing transportation networks and improving safety, efficiency, and sustainability. This
    research aims to advance the development of ITS through the utilization of Multi-agent
    Systems (MAS) in vehicle simulator software. The opportunities and underlying challenges of
    successful  simulation of an ITS system in a vehicle simulator software are investigated
    and consequentially, a framework for ITS solution implementation into a simulator software
    is presented. The agent-based framework is able to facilitate simulating complex behaviour
    of various road users as well as road equipment such as (cooperative) traffic lights. The
    main advantage of the framework is modularization of agent's skills which makes it easier
    to model complex behaviour. The framework also simplifies implementation of communication
    interface to arbitrary agent, which makes it a strong tool for Cooperative ITS (C-ITS)
    simulation. Finally, a simple ITS solution featuring a connected traffic light which
    provides information to connected vehicles for dynamic navigation is successfully
    integrated into the simulator using the framework. The framework can enable researchers to
    be able to implement various Intelligent Transport systems into vehicle simulator software
    while reducing development time and increasing maintainability of the software.
    \end{abstract}

\tableofcontents
\listoftables
\listoffigures
\newpage

\section{Introduction}

This thesis aims to investigate the possibilities and delimitations of multi-agent systems
application(s) for Intelligent Transport Systems (ITS) modeling in an IVS. Simulating traffic
behavior in a virtual environment, such as an Interactive Vehicle Simulator (IVS), is a complex
problem, mainly because of its highly dynamic nature. Trying to accurately simulate traffic on
a microscopic level, where individual road users (e.g. vehicles) are the elementary units in a
system requires complex modelling. All vehicles need to interact with each other and act upon
other drivers' actions. Because agent-based simulations have proven to model complex behavior
well \cite{Shoham}, this modeling technique seems like a suitable solution for achieving a realistic traffic
environment for IVS. 

In the first two chapters, a review of the state-of-art research on Intelligent Transport
Systems and Multi-agent Systems (MAS) are presented and important features related
to the main topic of the thesis are discussed. This research leads to the definition of system requirements
that are described in the third chapter. In the fourth section, a framework architecture 
is developed, in line with MAS paradigms for simulation of ITS solutions. In the fifth
section, development tools and their delimitations for framework development are presented.
The simulator software integration, additional development platforms and
libraries facilitating the framework development process are discussed. In the sixth section,
a technical implementation of the framework is presented, showing the implementation details
while also serving as the software's documentation. In section number seven, a validation of
the proposed framework is conducted by implementing a distributed cooperative ITS system into the
IVS, using the proposed framework and assessing its performance and overall results.

\subsection{Introduction to the problem}

In a world where vehicle transport plays an inseparable role in the society, with an ever-increasing
demand, it is important to analyze and study driver behavior and inherently interactions and
relationships between drivers and their environment. Even though
substantial advancements in autonomous driving are being made, for the near future, vehicles 
will be controlled by humans. Consequentially, this makes them exposed to a substantial danger. 
Data shows that about 95 \% of traffic accidents are a result of human error \cite{Parliament2021}. 
Each traffic accident has got a tremendous effect on socioeconomic growth. A study by the European
Union states that accident-related expenses (including the cost of fatal accidents) cost 1,8 \% of the EU's GDP \cite{Wijnen2017}.  
A rather non-cynical point of view is that each life lost is a failure in the society itself.
Therefore, an effort should be made to eliminate fatal accidents.

A method that has proven to be effective at studying driver behavior and traffic safety is 
performing research through the usage of an interactive vehicle simulator (IVS), which allows to undertake experiments in a safe,
controlled and reproducible environment \cite{Winter2012}. Because the vehicle simulator is a digital
twin of a real vehicle, it is desirable to make the interaction between the driver
and IVS as close to reality as possible. Making such effort naturally improves the trustworthiness of
acquired data from experiments and potentially also range of IVS applications. The IVS has
got a broad spectrum of utilization. It is not only used as a tool to research driver
behaviour, but also used in development and testing of advanced driver-assistance systems,
extending the simulator to a hardware-in-the-loop or vehicle-in-the-loop software, which
enables testing real hardware in virtualized traffic \cite{Horvath2019}.

\subsection{Aim}

The goal of this thesis is to investigate multi-agent systems (MAS) and evaluate possibility of
application of these systems in the scope of Intelligent Transport Systems (ITS) simulation.
The objective is to find out whether the shared characteristics of MAS \& ITS, e.g.
distribution of intelligence, make agent-based modeling a strong tool for simulating ITS
solutions. The output of the experimental/practical part of the thesis is a MAS-based
simulation framework for facilitating the process of ITS implementation into an IVS software with the benefit
of modular and easy-to-extend functionality. The developed system should also be able to model
state-of-the-art ITS systems that utilize communication between road users, such as C-ITS
systems. 

 \subsection{Research questions}

Given the main goal of the thesis, which is to empirically investigate feasibility of ITS implementation into 
simulation software, the research questions are defined so that they help understand the scope of the thesis and 
expected outcomes. 

% \begin{itemize}
%     \item Agent-based modeling and multi-agent systems introduction and application examples
%     \item Research of possible application of MAS or ABM for simulation of ITS systems
%     \item Research of available modules and libraries for implementation of MAS in different programming languages
%     \item Propose and design a methodology and algorithms for implementation for the simulation of an ITS into an existing IVS using ABM
%     \item Evaluate advantages and disadvantages of ABM in IVS for ITS simulation
% \end{itemize}

\begin{itemize}
    \item Can agent-based modeling be applied to simulate Intelligent Transport Systems in vehicle simulators and are they a viable option?
    \item How would an ITS simulation be implemented into an existing IVS using agent-based modelling?
    \item What are the available modules and libraries for implementing MAS in different programming languages?
    \item What are the advantages and challenges associated with using agent-based modeling to simulate Intelligent Transport Systems in vehicle simulators, and how can they be addressed?
\end{itemize}

% \begin{itemize}
% \end{itemize}

\subsection{Methodology}

The main objective of the thesis is to create a framework for ITS implementation into a vehicle simulator software. 
In the first step, a literature review regarding research topics of the thesis is performed - Multi-agent Systems and 
Intelligent Transport Systems. The idea is to gain an insight into how to approach modeling systems using 
MAS. Secondly, research on the current state of Intelligent Transport Systems is conducted in 
order to identify system features and interactions that are important when attempting to create a 
base model for ITS implementation. 

Based on the analyzed resources, requirements for the framework's functionality are defined.
With the framework's requirements in consideration, implementation options and possibilities of
external software usage are discussed.  With the implementation scope defined, architecture of
the simulation framework is presented, including component description and its internal structure.
After the framework's structure defined, integration with simulator software is discussed. 
With the completed framework definition, implementation using a chosen software/programming language
is done through definition of individual components and their interface. 

The the end goal is to implement a module/framework for general ITS implementation in a vehicle simulator. 
The main objective of the framework development will be to create a framework that will be sufficiently modular for 
a wide variety of ITS solutions to implement. At the same time, it should be easy and
straightforward to use. The framework is evaluated by implementing a sample ITS simulation,
which helps to empirically evaluate the implementation process and the framework's capabilities
and shortcomings.

\subsection{Delimitations}

Research delimitations refer to the scope of the study. The delimitations for a thesis on
Agent-based modeling of Intelligent Transport Systems in vehicle simulator include mainly the
technical scope. The framework needs to be compatible with the Unity game development
engine which is used in the IVS laboratories of the Czech Technical University University in
Prague (CTU). It should be well-integrated with the simulator software so that the development
using the framework offers seamless interaction with other system components. 

\subfile{1its}

\subfile{2mas}

\subfile{3requirements}

\subfile{4toolset}

\subfile{5system}

\subfile{6implementation}

\subfile{7validation}

\subfile{8discussion.tex}

\clearpage

\section{Conclusion}

ITS solutions are often systems integrating a large number of 
actors in a highly dynamic environment. In recent times, modern engineering made it possible 
to equip mobile devices, including road vehicles, with high-performance computers, making 
distributed intelligence a promising field for research, including Intelligent Transport Systems 
research.

The goal of this thesis was to investigate Multi-Agent Systems, which is a sub-field of Artificial Intelligence 
research, especially research related to the application of Multi-Agent Systems in Intelligent Transport Systems 
simulation. 

First, the area of Intelligent Transport Systems was investigated, setting delimitations and important features 
that were important to the topic of the thesis. The relationship between IVS and ITS systems was discussed, leading 
to a discussion about intelligence distribution and current research in the ITS field. 

The second part was dedicated to MAS literature review and the main principles of MAS were discussed. Furthermore, MAS 
individual MAS architectures were described and reviewed. Finally, the concepts of communication between agents were investigated,
defining requirements on how the agents should communicate. 

In the practical part of the thesis, delimitations and system requirements were defined first in order to facilitate the 
process of implementation and clearly define the system's capabilities. The development platform was briefly discussed, 
going over the options for potential facilitation of the implementation of the proposed system. It was identified that 
there are no suitable modules for MAS implementation to an IVS, as the research modules were either outdated or focused on 
different aspects of multi-agent systems. Therefore, it was decided that the framework would be built from the ground up.

Next, the actual system was proposed. The system design was divided into two parts - a
micro-architecture that focused on the inner structure of individual agents, whose goal was to
make the agents sufficiently modular as well as to actually facilitate the ITS implementation.
The architecture was designed based on the literature review in the preceding chapters. The
second part was dedicated to the macro architecture, which described how the individual agents
should interact with each other. The integral part of agent interaction was communication
specification, how agents will send and receive information.  Based on the preceding research
on ITS and MAS, the ETSI messaging standards (CAM \& DENM) were decided to be implemented.

After the system for MAS-based ITS systems implementation has been specified, the next part was
devoted to the system implementation.  The outcome is a highly modular framework that is
integrated with the chosen simulator software, with general behavior specific to MAS \& ABM 
and the architecture pre-implemented. The framework was implemented using the C\# programming 
language and the simulator engine (Unity) API. 

The developed framework was then validated by choosing a case appropriate for demonstration of the 
framework's utility. The case was chosen to be a dynamic routing system whose purpose was to 
improve traffic conditions by harmonizing a road network's load by the use of cooperative 
traffic lights and connected vehicles. The system was successfully implemented and the analysis of the 
simulation showed that the system is behaving as expected, i.e. reducing the overall travel time when 
calibrated.

Overall, the goals specified by the thesis guidelines have been fulfilled. The outcome of the thesis is a 
modular C\# framework/module which facilitates implementation of various (Cooperative) 
Intelligent Transport Systems solutions implementation into a vehicle simulator software. 

\clearpage

\thispagestyle{empty}

\clearpage

\printbibliography
\end{document}
