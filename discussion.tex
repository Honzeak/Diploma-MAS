\documentclass[main.tex]{subfiles}
\graphicspath{{./img}}
\begin{document}
\section{Discussion}

This section will go over the overall process the work that has been done regarding the thesis and evaluate the outcomes.
Firstly, it should be assessed whether the thesis has fulfilled the given guidelines in the thesis assignment. The following 
section will go over the guidelines and discuss whether the assignment outline has been fulfilled. 

\ul{\textbf{Guideline \#1:} Agent based modeling and multi-agent systems introduction and application examples}

In section (\ref{sec-mas}), the literature regarding Multi-agent Systems was reviewed. The review is focused mainly on the 
primary characteristics of MAS and on various types and implementation methodologies of MAS. The outcome of the section 
was identifying important implementation details of MAS which aided the ITS implementation in the practical part. 

\ul{\textbf{Guideline \#2:} Research of possible application of MAS or ABM for simulation of ITS systems}

In section (\ref{sec-its}), the current research regarding ITS systems was reviewed. The current trends in ITS suggest that 
modern ITS systems will rely on maximizing interoperability between individual road users, utilizing direct, non-centralized wireless 
communication. Such features are common in agent-based systems as well, therefore it has been concluded that MAS could represent 
a convenient way how to simulate such systems. Cooperative ITS (C-ITS) systems and connected In-vehicle Informatio Systems were 
identified as the most suitable systems for application using MAS.

\ul{\textbf{Guideline \#3:} Research of available modules and libraries for implementation of MAS in different programming
languages}

As a secondary objective to building an ITS simulation framework for an IVS, it was required to review available MAS implementation 
modules that would facilitate building the desired framework. Multiple modules were reviewed, including their main use-cases 
and implementation details. After discussing the main aspects of integration into the simulator
software (section (\ref{sec-toolset})), it was concluded that there are no suitable modules for MAS implementation to the
specific IVS, as the researched modules were either outdated or focused on a different aspects
of multi-agent systems. Therefore, it was decided that the framework would be built from the
ground up.

\ul{\textbf{Guideline \#4:} Propose and design a methodology and algorithms for implementation for the 
simulation of an ITS into an existing IVS using ABM.}

In section (\ref{sec-system}), a system to integrate novel ITS solutions into a vehicle simulator was developed using the research findings in the 
preceding parts of the thesis. The system implemented the \textsubscript{3}T architecture for individual agents that was supposed 
to offer sufficient balance between system complexity and modularity. The architecture of the macro-system, which defined mainly 
how agents interact between each other, has been also developed in line with requirements based on MAS research as well. The 
system was then implemented using the C\# programming language, because of its object-oriented properties and the fact that 
the intended simulator software uses C\# scripting API as well, allowing for deep integration with the software. The implementation comprised of partially functional classes that 
defined the system's structure but were open to detailed logic implementation based on the specific use-case. 

\ul{\textbf{Guideline \#5:} Evaluate advantages and disadvantages of ABM in IVS for ITS simulation}

The system evaluation was, in part, done by implementing a sample C-ITS system into the simulator software, with cooperative 
traffic light and connected vehicles. The implementation of the sample system has proved that it is possible to implement specific 
ITS solutions using the developed framework, therefore simulating Intelligent Transport Systems using agent-based modelling is 
possible.

\subsection{Advantages}

This section will go over the identified advantages of using 

\end{document}