\documentclass[main.tex]{subfiles}
\graphicspath{{./img}}

\begin{document}
\section{System validation}

This section will be devoted to evaluation and demonstration of the proposed MAS-based ITS
implementation framework. A ITS solution will be chosen to demonstrate implementation using 
the framework, serving as a validation of the system and the overall implementation process will 
be conducted, as well as demonstration of the framework's capabilities. 

The following paragraphs will pick up on the findings in chapter (\ref{sec-croads}), where the
table (\ref{gdt-mapping}) maps General Driving Tasks to state-of-the-art ITS solutions. Those
ITS solutions will be analyzed in this chapter in an attempt to transparently select an optimal
system to implement using the developed framework in order to validate and demonstrate its
use-case. 

In order to choose an optimal ITS to implement, it is important to define features that 
reflect such system's suitability for implementation. Optimally selecting such features 
for evaluation of previously selected individual ITSs will create a qualitative benchmark that
will be used to select an ITS solution for demonstration of the proposed framework's
capabilities.

\subsection{Qualitative analysis}

The method to determine how well-suited the candidate system type is for the implementation to
an IVS will be to evaluate it based on the \emph{three} features discussed in the preceding
section (\ref{sec-its}): 

\begin{itemize}
    \itemspacing{.5}
    \item Intelligence distribution rate
    \item Driver engagement
    \item Research relevance
\end{itemize}

A set of integer values will determine the rates how the selected features are represented in the 
ITS solution in question. The representation level will be expressed in a fuzzy logic given in four levels:

\begin{itemize}
    \itemspacing{.5}
    \item \talign{None}{5}{$\rightarrow$\quad 0}
    \item \talign{Low}{5}{$\rightarrow$\quad 2}
    \item \talign{Moderate}{5}{$\rightarrow$\quad 5}
    \item \talign{High}{5}{$\rightarrow$\quad 8}
\end{itemize}

Because \emph{None} representation of the feature does not bear any significance to choosing an ITS 
in question, it will be assigned zero. \emph{Low} representation should not have a significant impact on 
suitability of ITS, therefore its value is only incremented by two as opposed to \emph{None}. \emph{Moderate}
and \emph{High} rates of feature representation should have a more significant influence on the final system 
choice, therefore their values have been assigned as increments of three instead of two. 

Each system reviewed in section (\ref{sec-its}) will be evaluated and the sum of individual feature evaluation 
will determine its score, meaning the higher the score, the better the result for a particular ITS. The system 
with the \emph{highest} score will be chosen to implement using the proposed framework. The results of the 
qualitative analysis can be seen in the table (\ref{qa-table}) below.

\begin{table}[htbp]
    \caption{Quantitative analysis results}
    \renewcommand{\arraystretch}{1.4}
    \centering\begin{tabular}{l*{3}{r}r} \toprule
         & \multicolumn{3}{c}{Features} & \\ \cmidrule(rl){2-4}
        ITS & \multicolumn{1}{p{6em}}{MAS-compatibility} & \multicolumn{1}{p{6em}}{Driver \newline engagement} & \multicolumn{1}{p{6em}}{Research \newline relevance} & Score \\ \midrule
        E-Call & 2 & 2 & 5 & 9 \\ 
        Electronic fee collection & 2 & 0 & 2 & 4 \\
        Parking guidance & 2 & 8 & 5 & 15 \\
        Network traffic control & 8 & 0 & 8 & 16 \\
        \textbf{Cooperative ACC} & 8 & 8 & 8 & \textbf{24} \\
        European Truck Platooning & 5 & 2 & 8 & 15 \\
        ADASIS & 2 & 8 & 8 & 18 \\
        Mobility as a Service & 8 & 2 & 8 & 18 \\
        Map services (Geo-fencing) & 2 & 5 & 8 & 15 \\
        \multicolumn{1}{p{5em}}{Self-driving algorithms} \& Platooning & 8 & 2 & 8 & 18 \\
        \textbf{IVIS} & 8 & 8 & 8 & \textbf{24} \\ \bottomrule
    \end{tabular}
    \label{qa-table}
\end{table}

A methodology for qualitative analysis was introduced and used to
determine the best ITS candidates for implementation into an IVS. The resulting best fitted
systems were both from the C-ITS field, as its features are much alike those of multi-agent
systems. Finally, the chosen systems to implement were the \emph{IVIS based awareness and
warning information system} and the \emph{Green Light Optimal Speed Advisory} system. The
following practical part will be dedicated to the implementation methodology, introduction to
the development system and the development itself.

\clearpage 

\end{document}