\documentclass[0main.tex]{subfiles}

\graphicspath{{./img}}
\begin{document}

\section{Requirements definition}\label{sec-requirements}

This section is mainly devoted to defining what should the proposed system be capable of in
terms of functionality. The previous sections are devoted to discussion about the main ITS
simulation aspects of the proposed framework. This section expands on it, going into more
details and processes on how to achieve a system that is both modular enough to support a wide
range of ITS solution simulation and offers enough facilitation value for streamlining the
development of ITS simulations at the same time. 

Another outcome of this section is to determine which toolset to use to facilitate the building of 
the framework. Ideally, an existing library for agent-based modeling should be used, which has got 
its structure in line with the proposed specifications in this section.

Furthermore, regarding the topic of this thesis, the inductive approach of analyzing actors and
processes of various ITS solutions is then used to validate the proposed framework. The
framework is designed using the obtained knowledge in the literature review researched in the
previous section \ref{sec-mas}. Consequentially, an ITS solution is chosen to implement
using the proposed framework, which will serve both as a proof-of-concept and as a benchmark
for evaluation of the framework.  

\subsection{System requirements}

Defining requirements ensures that functionality identified in both the inductive system
decomposition in section \ref{sec-its}, and the deductive approach in section \ref{sec-mas}
are met. This results in an ITS virtual-deployment framework that is highly modular and
capable of simulating a large number of ITS. The requirements are summarized in the table
\ref{sys-requirements}.

\begin{table}[htbp]
    \small
    \caption{Proposed system requirements}
    \centering\begin{tabular}{>{\footnotesize}p{.9\textwidth}}
        \toprule 
\textbf{Requirement 1}: The framework shall be a multi-agent based system, supporting more actors 
that can act independently, have their logic and can make decisions based on their 
own perception of the environment.
\\ \midrule
\textbf{Requirement 2}: The framework shall support modular architecture by employing a layer-based 
architecture model
\\ \midrule
\textbf{Requirement 3}: The implementation of a particular ITS agent's elementary
capability shall be done using skill modules, each of them serving one purpose. The skill modules must provide a
pre-defined input and output interface.
\\ \midrule
\textbf{Requirement 4}: An agent shall be able to communicate with others through a dedicated communication module/skill 
that will manage the communication.
\\ \midrule
\textbf{Requirement 5}: Agents shall be able to act upon received information from their sensory and communication interfaces
and dynamically adjust their plans using the deliberation and sequencing layer.
\\ \midrule
\textbf{Requirement 6}: Sensory and communication capability of an agent shall be incorporated 
as a pre-implemented module.
\\ \midrule
\textbf{Requirement 7}: Where applicable, agents shall be able to detect conflicting intentions with other agents
while executing their tasks. 
\\ \midrule
\textbf{Requirement 8}: Agents shall resolve conflicts using a standardized communication
specification, and comply with its requirements, mentioned in section \ref{sec-acl}. An
interface to support basic negotiations shall be provided.
\\ \midrule
\textbf{Requirement 9}: The framework's architecture shall be modular enough to support a broad 
spectrum of ITS, including but not exclusive to C-ITS solutions.
\\ \midrule
\textbf{Requirement 10}: The framework shall support C-ITS messaging services (Decentralized
Environmental Notification Basic Service (DENM) \& Cooperative Awareness Basic Service (CAM))
out-of-the-box and according to their specifications.
\\ \midrule
\textbf{Requirement 11}: The supported communication modes shall be both direct (messaging specific agents) and indirect  
(broadcast \& subscription). 
\\ \bottomrule
    \end{tabular}
    \label{sys-requirements}
\end{table}

\clearpage

\end{document}