\documentclass[main.tex]{subfiles}

\graphicspath{{./img}}
\begin{document}

\section{Requirements definition}\label{sec-requirements}

This section will be mainly devoted to defining what should the proposed system be capable of in terms 
of functionality. In the previous section, general guidelines on which ITS solution the proposed 
framework should aim to be able to simulate were discussed. This section will expand on it, 
going into more details and processes on how to achieve a system that is both modular enough to support 
a wide range of ITS solution simulation and, at the same time, offering enough facilitation value 
for streamlining the development of ITS simulations. 

Another outcome of this section will be to determine which toolset to use to facilitate the building of 
the architecture. Ideally, an existing library for agent-based modeling could be used, which has got 
the elementary structure defined according to our proposed specifications.

Furthermore, regarding the topic of this thesis, the inductive approach of analyzing actors and
processes of Various ITS solutions will then be used to validate the proposed framework. The
framework will be designed using knowledge (e.g. optimal agent architecture) researched in the
past section (\ref{sec-mas}). Consequentially, an ITS solution will be chosen to implement
using the proposed framework, which will serve both as a proof-of-concept and as a benchmark
for evaluation of the framework.  

\subsection{System requirements}

As has been stated in the introduction, this section will be dedicated to proposing logical
requirements for the system architecture and model. Defining such requirements will ensure that
functionality identified in both the inductive system decomposition in section (\ref{sec-its}),
i.e. the ITS solutions review, and the deductive approach in section (\ref{sec-mas}), which 
outlines the basics of how multi-agent systems are modeled, will be met. This should result 
in an ITS virtual-deployment framework that is highly modular and capable of simulating a large 
number of ITS. The requirements are summarized in the table (\ref{sys-requirements}) below.

\begin{table}[htbp]
    \small
    \caption{Proposed system requirements}
    \centering\begin{tabular}{>{\footnotesize}p{.9\textwidth}}
        \toprule 
\textbf{Requirement 1}: The framework should be a multi-agent based system, supporting more actors 
that can act independently, have their logic and can make decisions based on their 
own perception of the environment.
\\ \midrule
\textbf{Requirement 2}: The framework should support modular architecture by employing a layer-based 
architecture model
\\ \midrule
\textbf{Requirement 3}: The implementation of a particular ITS agent's elementary
capability should be done using skill modules, each of them serving one purpose. The skill modules must provide a
pre-defined input and output interface.
\\ \midrule
\textbf{Requirement 4}: An agent should be able to communicate with others through a dedicated communication module/skill 
that will manage the communication.
\\ \midrule
\textbf{Requirement 5}: Agents should be able to act upon received information from their sensory and communication interfaces
and dynamically adjust their plans using the deliberation and sequencing layer.
\\ \midrule
\textbf{Requirement 6}: Sensory and communication capability of an agent should be incorporated 
as a pre-implemented module.
\\ \midrule
\textbf{Requirement 7}: Where applicable, agents should be able to detect conflicting intentions with other agents
while executing their tasks. 
\\ \midrule
\textbf{Requirement 8}: Agents should resolve conflicts using a standardized communication
specification, and comply with its requirements, mentioned in section \ref{sec-acl}. An
interface to support basic negotiations should be provided.
\\ \midrule
\textbf{Requirement 9}: The framework's architecture should be modular enough to support broad 
a spectrum of ITS implementation, including but not exclusive to C-ITS solutions.
\\ \midrule
\textbf{Requirement 10}: The framework should support C-ITS messaging services (Decentralized
Environmental Notification Basic Service (DENM) \& Cooperative Awareness Basic Service (CAM))
out-of-the-box and according to their specifications.
\\ \midrule
\textbf{Requirement 11}: The supported communication modes should be both direct (messaging specific agents) and indirect  
(broadcast \& subscription). 
\\ \bottomrule
    \end{tabular}
    \label{sys-requirements}
\end{table}

\clearpage

\end{document}