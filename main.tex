\documentclass[titlepage, 12pt]{article}

\title{
\includegraphics[width=.3\textwidth]{cvut-logo.jpg}\par
\vspace{10mm}
\indent
\textbf{CZECH TECHNICAL UNIVERSITY IN PRAGUE}

FACULTY OF TRANSPORT SCIENCES

\vfill

{\Large JAN MACEK}
\vspace{10mm}

AGENT-BASED MODELING OF ITS SYSTEMS IN VEHICLE SIMULATOR 
\vspace{15mm}

{\Large MASTER'S THESIS}
\vfill

}
\date{\Large 2022}

\usepackage[utf8]{inputenc}
\usepackage[T1]{fontenc}
% \usepackage{cm-super}
\usepackage{listings}
\usepackage{fancyhdr}
\usepackage{graphicx}
\usepackage{parskip}
\usepackage{svg}
\usepackage[a4paper, top=25mm, bottom=25mm, left=30mm, right=20mm, twoside]{geometry}
\usepackage{tabularx}
\usepackage{caption}
\usepackage{hyperref}
\usepackage{xcolor}
\usepackage{amsmath}
\usepackage[sorting=none]{biblatex} 
\usepackage{color}
\usepackage{pdfpages}
\usepackage{enumitem}
\usepackage{amssymb}
\usepackage{booktabs}
\usepackage{microtype}
\usepackage{subfiles}
\usepackage{multirow}
\usepackage{pdfpages}
\usepackage{enumitem}
\usepackage{soul}

\definecolor{dkgreen}{rgb}{0,0.6,0}
\definecolor{gray}{rgb}{0.5,0.5,0.5}
\definecolor{mauve}{rgb}{0.58,0,0.82}
\definecolor{inline}{rgb}{0.25,0.25,0.80}


\addbibresource{references.bib}

\hypersetup{
    colorlinks,
    linkcolor={red!50!black},
    citecolor={blue!50!black},
    urlcolor={blue!70!black}
}

\pagestyle{fancy}

\graphicspath{{./img}}

\fancyhead[L]{Jan Macek}
\setlength{\headheight}{15pt}

\newlength{\varSepscale}
\setlength{\varSepscale}{16pt}
\newcommand{\itemspacing}[1]{\setlength\itemsep{\dimexpr #1\varSepscale-\varSepscale}}
\newcommand{\term}[1]{\hspace*{2em}\textit{#1}\, -\,}
\newcommand{\rot}[1]{\rotatebox[origin=c]{90}{#1}}
\newcommand{\talign}[3]{\makebox[#2em]{#1\hfill}#3}

%DOCUMENT
\begin{document}
\setlength{\baselineskip}{1.5em}
\maketitle

\includepdf[pages=-]{zadani-dp-jan-macek.pdf}

\begin{abstract}
    TODO
\end{abstract}

\tableofcontents
\newpage

\section{Introduction}

The goal of this thesis is to investigate multi-agent systems (MAS) and their principles and evaluate the usages of these systems related to ITS research, where 
their shared characteristics of distributed interoperability make
agent-based modelling a strong tool for simulating ITS solutions. The output of 
the experimental/practical part of the thesis is a MAS-based simulation framework 
for facilitating the process of ITS solutions simulation with the benefit 
of modular and fast extensibility.

In the first two chapters, a review of the state-of-art research on Intelligent 
Transport Systems and Multi-agent Systems are presented, discussing the important 
features related to the main topic of the thesis. This research leads to the definition of system
requirements that are described in the third chapter. In the 
fourth section, a framework architecture model is proposed, in line with 
MAS paradigms and adapted for ITS solutions simulation. In the fifth section, the
usage and requirements of tools for framework development are discussed, mainly describing the simulator software integration
and additional development platforms and libraries that could potentially facilitate the
framework design process. In the sixth section, a technical implementation of the 
framework is presented, showing the implementation details while also serving as the 
software's documentation. In section number seven, a validation of the proposed 
system is conducted by implementing a distributed cooperative ITS system into the
IVS, using the proposed framework and assessing its performance and overall results.

\subsection{Introduction to the problematic}

In a world where vehicle transport plays an inseparable role in society, with an ever-increasing
demand, it is important to analyze and study driver behaviour and inherently interactions and
relationships between drivers and their surroundings. This is supported by the fact that, even though
substantial advancements in autonomous driving are being made, for the near future, vehicles 
will be controlled by humans and therefore be exposed to a substantial risk of danger. 
Data shows that about 95 \% of traffic accidents are a result of human error \cite{Parliament2021}. 
Each traffic accident has got a tremendous effect on socioeconomic growth. A study by the European
Union states that accident-related expenses (including cost of fatality) cost 1,8 \% of EU GDP \cite{Wijnen2017}.  
A rather non-cynical point of view is that each life lost is a failure in society itself and an effort
should be made to diminish fatal accidents.


A method that has proven to be effective at studying driver behaviour and
traffic safety is by using interactive vehicle simulators (IVS), which allow to
undertake experiments in a safe, controlled and reproducible way \cite{Winter2012}.  Because
the driving simulator is a digital twin of a real vehicle, it is naturally
reasonable to make the interaction between the driver and IVS as close to
reality as possible, which inherently improves the trustworthiness of acquired data from experiments
and potentially also the range of IVS applications. The IVS has got a broad
spectrum of utilization. It is not only used as a tool to research driver
behaviour, but also used in development and testing of advanced
driver-assistance systems, extending the simulator to a hardware-in-the-loop or
a vehicle-in-the-loop software, which enables to test real hardware and simulate
full road testing \cite{Horvath2019}.

\subsection{Aim}

Simulating the traffic environment is a complex problem, mainly because of
its highly dynamic characteristics. All vehicles need to interact with
each other and act upon other drivers' actions. Because agent-based simulations
have proven to model complex behaviour well, this modelling technique seems like
a suitable solution for achieving a realistic traffic environment for IVS. The 
aim of this thesis is to investigate the possibilities and delimitations of 
multi-agent systems application(s) for ITS modelling in an IVS. 

\subsection{Research questions}

In order to evaluate the research, the the following questions are given that 
the thesis should answer, which are following up on the thesis assignment. 

\begin{itemize}
    \item How can agent-based modeling be applied to simulate Intelligent Transport Systems in vehicle simulators?
    \item What are the key features that need to be included in an agent-based model to accurately represent Intelligent Transport Systems in vehicle simulators?
    \item Is agent-based modelling a suitable tool for application of ITS in a vehicle simulator?
    \item What are the limitations and challenges associated with using agent-based modeling to simulate Intelligent Transport Systems in vehicle simulators, and how can they be addressed?
\end{itemize}

\subsection{Methodology}

The methodology for the IVS-based MAS research will include the following: 

\begin{enumerate}
    \item Agent-based modeling and multi-agent systems introduction
    \item Research of possible applications of MAS or ABM simulation of ITS
    \item Requirements specification for the implementation of MAS in IVS
    \item Research of available modules for implementation of MAS
    \item Propose and design a framework for ITS implementation into and existing IVS using ABM
    \item Evaluate the system and its advantages and disadvantages
\end{enumerate}

\subsection{Delimitations}

Research delimitations refer to the scope of the study. The delimitations
for a thesis on Agent-based modeling of Intelligent Transport Systems in vehicle simulator include mainly 
the technical scope. The framework will need to be compatible with the Unity game development engine which 
is used in the faculty's IVS laboratories. 

The developed system should also be able to model state-of-the-art ITS systems that utilize communication between 
road users, such as C-ITS systems. 



%research question -  Find a way to simulate its in using its | ze zadani
% delmitation - software
% methodology - vsechno co jsme delali - prozkoumam mas, prozkoumam existujici methodology


\subfile{1its}

\subfile{2mas}

\subfile{3requirements}

\subfile{4toolset}

\subfile{5system}

\subfile{6implementation}

\subfile{7validation}

\subfile{8discussion.tex}

\clearpage

\section{Conclusion}

ITS solutions are often systems integrating a large number of 
actors in a highly dynamic environment. In recent times, modern engineering made it possible 
to equip mobile devices, including road vehicles, with high-performance computers, making 
distributed intelligence a promising field for research, including Intelligent Transport Systems 
research.

The goal of this thesis was to investigate Multi-Agent Systems, which is a sub-field of Artificial Intelligence 
research, especially research related to the application of Multi-Agent Systems in Intelligent Transport Systems 
simulation. 

First, the area of Intelligent Transport Systems was investigated, setting delimitations and important features 
that were important to the topic of the thesis. The relationship between IVS and ITS systems was discussed, leading 
to a discussion about intelligence distribution and current research in the ITS field. 

The second part was dedicated to MAS literature review and the main principles of MAS were discussed. Furthermore, MAS 
individual MAS architectures were described and reviewed. Finally, the concepts of communication between agents were investigated,
defining requirements on how the agents should communicate. 

In the practical part of the thesis, delimitations and system requirements were defined first in order to facilitate the 
process of implementation and clearly define the system's capabilities. The development platform was briefly discussed, 
going over the options for potential facilitation of the implementation of the proposed system. It was identified that 
there are no suitable modules for MAS implementation to an IVS, as the research modules were either outdated or focused on 
a different aspects of multi-agent systems. Therefore, it was decided that the framework would be built from the ground up.

Next, the actual system was proposed. The system design was divided into two part- a micro
architecture that focused on the inner structure of individual agents, whose goal was to make
the agents sufficiently modular as well as to actually facilitate the ITS implementation. The
architecture was designed based on the literature review in the preceding chapters. The second
part was dedicated to the macro architecture, which described how the individual agents should
interact between each other. The integral part of agent interaction was communication
specification, how agents will send and receive information.  Based on the preceding research
on ITS and MAS, the ETSI messaging standards (CAM \& DENM) were decided to be implemented.

After the system for MAS-based ITS systems implementation has been specified, the next part was
devoted to the system implementation.  The outcome is a highly modular framework that is
integrated with the chosen simulator software, with general behaviour specific to MAS \& ABM 
and the architecture pre-implemented. The framework was implemented using the C\# programming 
language and the simulator engine (Unity) API. 

The developed framework was then validated by choosing a case appropriate for demonstration of the 
framework's utility. The case was chosen to be dynamic routing problem whose purpose was to 
improve the traffic conditions by harmonizing a road network's load by the use of cooperative 
traffic light and connected vehicles. The system was successfully implemented and the analysis of the 
simulation showed that the system is behaving as expected, i.e. reducing the overall travel time when 
calibrated.


\clearpage

\thispagestyle{empty}
\listoftables
\listoffigures
\clearpage

\printbibliography
\end{document}
