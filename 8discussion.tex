\documentclass[main.tex]{subfiles}
\graphicspath{{./img}}
\begin{document}
\section{Discussion}

This section will go over the overall process the work that has been done regarding the thesis and evaluate the outcomes.
Firstly, it should be assessed whether the thesis has fulfilled the given guidelines in the thesis assignment. The following 
section will go over the guidelines and discuss whether the assignment outline has been fulfilled. 

\ul{\textbf{Guideline \#1:} Agent based modeling and multi-agent systems introduction and application examples}

In section (\ref{sec-mas}), the literature regarding Multi-agent Systems was reviewed. The review is focused mainly on the 
primary characteristics of MAS and on various types and implementation methodologies of MAS. The outcome of the section 
was identifying important implementation details of MAS which aided the ITS implementation in the practical part. 

\ul{\textbf{Guideline \#2:} Research of possible application of MAS or ABM for simulation of ITS systems}

In section (\ref{sec-its}), the current research regarding ITS systems was reviewed. The current trends in ITS suggest that 
modern ITS systems will rely on maximizing interoperability between individual road users, utilizing direct, non-centralized wireless 
communication. Such features are common in agent-based systems as well, therefore it has been concluded that MAS could represent 
a convenient way how to simulate such systems. Cooperative ITS (C-ITS) systems and connected In-vehicle Information Systems were 
identified as the most suitable systems for application using MAS.

\ul{\textbf{Guideline \#3:} Research of available modules and libraries for implementation of MAS in different programming
languages}

As a secondary objective to building an ITS simulation framework for an IVS, it was required to review available MAS implementation 
modules that would facilitate building the desired framework. Multiple modules were reviewed, including their main use-cases 
and implementation details. After discussing the main aspects of integration into the simulator
software (section (\ref{sec-toolset})), it was concluded that there are no suitable modules for MAS implementation to the
specific IVS, as the researched modules were either outdated or focused on different aspects
of multi-agent systems. Therefore, it was decided that the framework would be built from the
ground up.

\ul{\textbf{Guideline \#4:} Propose and design a methodology and algorithms for implementation for the 
simulation of an ITS into an existing IVS using ABM.}

In section (\ref{sec-system}), a system to integrate novel ITS solutions into a vehicle simulator was developed using the research findings in the 
preceding parts of the thesis. The system implemented the \textsubscript{3}T architecture for individual agents that was supposed 
to offer a sufficient balance between system complexity and modularity. The architecture of the macro-system, which defined mainly 
how agents interact with each other, has been also developed in line with requirements based on MAS research as well. The 
system was then implemented using the C\# programming language, because of its object-oriented properties and the fact that 
the intended simulator software uses C\# scripting API as well, allowing for deep integration with the software. The implementation comprised partially functional classes that 
defined the system's structure but were open to detailed logic implementation based on the specific use case. 

\ul{\textbf{Guideline \#5:} Evaluate advantages and disadvantages of ABM in IVS for ITS simulation}

The system evaluation was, in part, done by implementing a sample C-ITS system into the simulator software, with a cooperative 
traffic light and connected vehicles. The implementation of the sample system has proved that it is possible to implement specific 
ITS solutions using the developed framework, therefore simulating Intelligent Transport Systems using agent-based modeling is 
possible.

\subsection{Results}

Based on the designed system and its implementation, the implications of the research activity and its benefits, caveats and application 
possibilities will be analyzed. 

\subsubsection{Benefits}

The main suspected benefit of developing an ITS implementation framework is obvious. In theory,
the framework should facilitate ITS implementation into a vehicle simulator so that it enables
the developer to focus more on the high-level implementation related to the ITS system itself,
as the structure of the underlying agent system would be pre-implemented. 

For instance, the developer implementing a given ITS solution into the IVS can implement a desired behavior of an actor in 
the system by dividing the algorithm into atomical skill modules which inherently makes developing complex behavior easier and more 
manageable, as well as open to extending its capabilities by iteratively adding more modules. Also, one of the greater benefits is 
the fact that communication API  for the system developer has been implemented, according to the C-ITS ETSI standards. This 
shifts the focus of implementation completely away from communication development. This has proven to be the case when developing 
dynamic vehicle navigation using C-ITS as part of the framework capabilities' validation, where
the focus was simply filling the pre-defined DENM and CAM messages with desired information
about the status of the system actors (e.g. connected vehicles). Given the inner architecture of an agent, they can perform a variety 
of different tasks based on their current state and the state of the environment which means that the actors can be developed 
to perform complex tasks with a wide variety of activities that can dynamically change. 

\subsubsection{Shortcomings}

When implementing the sample C-ITS system using the proposed framework, shortcomings were also
identified during the process. The perceived disadvantage of using the framework is that when
implementing a simple behavior that does not need complex decision-making, all the agent layers
need to be implemented, although they might not be fully utilized. For instance, the
\emph{Connected traffic light} in section (\ref{sec-implementation}) only has one goal in the
deliberation layer and also just one task to perform, which implies that the agent model could
have been reduced just to a static, cyclic execution of the skills used mainly for
communication with the connected vehicles. Therefore, the developed MAS framework is likely to
demonstrate its advantages only when building larger-scale ITS projects in the vehicle
simulator which require developing complex behavior logic for the agents. 

Another potential disadvantage that was identified during the framework development is that
although the 3-layer concept might be effective, it might not be intuitive to use for someone
new to the concepts for multi-agent systems. For instance, it is not very clear how to decide
on the atomicity of the individual actions the agent can perform. However, this problem can be
solved by providing a template for a "simple" agent which has got the deliberation (and
potentially also the middle sequencing layer) in a single state. 

\subsection{Future work \& implications}

Regarding future work and further improvements of the outcomes of this thesis, it would be
beneficial to implement agent negotiation, which has been discussed in the theoretical part of
the thesis but has not been implemented as part of the framework. The system validation section
has shown that it is possible to develop fully functional simulations of Intelligent Transport
Systems, however, developing a negotiation model for the agents would make the framework more
robust and would allow modeling systems with greater complexity.

Apart from the improvement mentioned above, the system can already be used to model ITS systems
with intelligent agents, including state-of-the-art cooperative ITS systems. To mention a few,
the framework can model various In-vehicle Information Systems, such as the Intersection
Collision Warning, Emergency Vehicle Warning or Overtaking Warning.  Apart from the mentioned
sensing \& perception systems, planning-focused ITS systems such as Dynamic Vehicle Routing or
Green Light Optimal Speed Advisory can be modeled. After the potential improvement of adding
agent negotiation, autonomous driving and related ADAS systems could be also possible to
implement using the framework.

\end{document}